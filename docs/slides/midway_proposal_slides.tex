\documentclass[aspectratio=169,11pt]{beamer}

% ---- Minimalist theme ----
\usetheme{default}
\usecolortheme{default}
\setbeamertemplate{navigation symbols}{}
\setbeamertemplate{frametitle continuation}{}
\setbeamertemplate{itemize items}[circle]
\setbeamertemplate{enumerate items}[default]

% ---- Font: TeX Gyre Heros ----
\usepackage[T1]{fontenc}
\usepackage{tgheros}
\renewcommand{\familydefault}{\sfdefault}
\usepackage{microtype}

% ---- Packages ----
\usepackage{amsmath}
\usepackage{booktabs}
\usepackage{array}
\usepackage{tikz}
\usepackage{xcolor}
\usepackage{hyperref}
\usetikzlibrary{arrows.meta,positioning,calc,fit}

% ---- Maastricht University colors ----
\definecolor{umdark}{HTML}{001C3D}
\definecolor{umorange}{HTML}{E84E10}
\definecolor{umlight}{HTML}{4A90C4}
\definecolor{umgray}{HTML}{6B7280}
\definecolor{umbg}{HTML}{F8F9FA}
\definecolor{umfaint}{HTML}{E5E7EB}

% ---- Apply colors to Beamer ----
\setbeamercolor{normal text}{fg=umdark}
\setbeamercolor{frametitle}{fg=umdark}
\setbeamercolor{title}{fg=umdark}
\setbeamercolor{subtitle}{fg=umgray}
\setbeamercolor{author}{fg=umgray}
\setbeamercolor{date}{fg=umgray}
\setbeamercolor{institute}{fg=umorange}
\setbeamercolor{itemize item}{fg=umorange}
\setbeamercolor{itemize subitem}{fg=umlight}
\setbeamercolor{enumerate item}{fg=umorange}
\setbeamercolor{block title}{bg=umdark,fg=white}
\setbeamercolor{block body}{bg=umdark!5,fg=umdark}
\setbeamercolor{block title alerted}{bg=umorange,fg=white}
\setbeamercolor{block body alerted}{bg=umorange!8,fg=umdark}
\setbeamercolor{block title example}{bg=umlight,fg=white}
\setbeamercolor{block body example}{bg=umlight!8,fg=umdark}

% ---- Frametitle style ----
\setbeamertemplate{frametitle}{%
  \vspace{0.35cm}%
  \noindent\hspace*{0pt}%
  \parbox{\textwidth}{%
    {\usebeamerfont{frametitle}\usebeamercolor[fg]{frametitle}\insertframetitle}%
    \vspace{2pt}\\%
    {\color{umorange}\rule{\textwidth}{1.2pt}}%
  }%
  \vspace{-2pt}%
}
\setbeamerfont{frametitle}{size=\large,series=\bfseries}
\setbeamersize{text margin left=0.7cm,text margin right=0.7cm}

% ---- Footline ----
\setbeamertemplate{footline}{%
  \hbox{%
    \begin{beamercolorbox}[wd=\paperwidth,ht=2.2ex,dp=1ex]{footline}%
      \hspace{1em}{\scriptsize\color{umgray}Midway Proposal -- Carrier Origin and Steganographic Detectability}%
      \hfill%
      {\scriptsize\color{umgray}\insertframenumber/\inserttotalframenumber}%
      \hspace{1em}%
    \end{beamercolorbox}%
  }%
}

% ---- Metadata ----
\title{Does the Source of Carrier Image Affect\\Steganographic Detectability?}
\subtitle{Full Midway Proposal}
\newcommand{\teamauthors}{Abdul Moiz Akbar \;|\; Malo Coquin \;|\; Daria Gjonbalaj \;|\; Nico Muller-Spath\\Jimena Naravaez del Cid \;|\; David Wicker \;|\; Nikolas Zouros}
\author[Project 2.2 Team]{Project 2.2 Team}
\institute{Department of Advanced Computing Sciences\\Maastricht University}
\date{Project 2.2 \;|\; February 2026}

\begin{document}

% ============================================================
% Title
% ============================================================
\begin{frame}[plain]
\vfill
\begin{center}
{\color{umorange}\rule{0.42\textwidth}{2pt}}\\[12pt]
{\LARGE\bfseries\color{umdark}\inserttitle}\\[10pt]
{\normalsize\color{umgray}\insertsubtitle}\\[16pt]
{\small\color{umdark}\teamauthors}\\[8pt]
{\small\color{umorange}\insertinstitute}\\[8pt]
{\footnotesize\color{umgray}\insertdate}\\[12pt]
{\color{umorange}\rule{0.42\textwidth}{2pt}}
\end{center}
\vfill
\end{frame}

% ============================================================
\begin{frame}{Agenda}
\begin{columns}[T]
\begin{column}{0.47\textwidth}
\begin{enumerate}
\item Motivation and problem statement
\item Research questions and hypotheses
\item Chosen approaches
\item Experiments and validation
\end{enumerate}
\end{column}
\begin{column}{0.47\textwidth}
\begin{enumerate}
\setcounter{enumi}{4}
\item Prototype status
\item Related work positioning
\item Relation to curriculum
\item Planning and passing requirements
\end{enumerate}
\end{column}
\end{columns}
\end{frame}

% ============================================================
\section{Motivation}
% ============================================================
\begin{frame}{Motivation and Problem Statement}
\begin{columns}[T]
\begin{column}{0.56\textwidth}
\small
\begin{itemize}
\item Image steganography detectability depends on carrier statistics, not only embedding logic (Petitcolas et al., 1999; Cheddad et al., 2010; Fridrich \& Kodovsky, 2012).
\item Most steganalysis benchmarks assume camera photos with familiar noise/compression traces.
\item Modern generators (Stable Diffusion, StyleGAN3) produce photorealistic images from different processes (Rombach et al., 2022; Karras et al., 2021).
\item Synthetic images have measurable statistical fingerprints that may alter detector behavior (Wang et al., 2020; Corvi et al., 2023).
\end{itemize}
\end{column}
\begin{column}{0.40\textwidth}
\begin{alertblock}{Central Problem}
\small
Do steganalysis methods designed and validated on photographs remain effective when the carrier is ML-generated?
\end{alertblock}

\begin{block}{Closest prior work}
\small
De et al. (2022) show AI-generated image steganography, but not a controlled real-vs-ML comparison with standardized LSB/DCT detectors.
\end{block}
\end{column}
\end{columns}
\end{frame}

\begin{frame}{Why This Study Matters}
\begin{columns}[T]
\begin{column}{0.32\textwidth}
\begin{block}{Security}
\small
\begin{itemize}
\item If synthetic carriers are harder to detect, attackers gain an easy evasion path.
\item If easier, defenders get a concrete screening advantage.
\end{itemize}
\end{block}
\end{column}
\begin{column}{0.32\textwidth}
\begin{block}{Scientific gap}
\small
\begin{itemize}
\item Interaction between generative-model distributions and embedding distortion is largely unexplored.
\item Need controlled experiments isolating carrier origin.
\end{itemize}
\end{block}
\end{column}
\begin{column}{0.32\textwidth}
\begin{block}{Practical relevance}
\small
\begin{itemize}
\item AI images are now common in social and communication channels.
\item Practitioners need evidence on retraining/adaptation requirements.
\end{itemize}
\end{block}
\end{column}
\end{columns}
\vspace{0.15cm}
{\small\color{umgray}\textbf{Study scope:} 2 x 2 x 3 x 2 factorial design over 1,000 images, with CPU-feasible execution in 7 weeks.}
\end{frame}

% ============================================================
\section{Research Questions}
% ============================================================
\begin{frame}{Research Questions}
\small
\begin{block}{RQ1: Carrier origin effect}
Does carrier origin (real vs ML-generated) change detectability under identical embedding settings?
\end{block}

\begin{block}{RQ2: Payload sensitivity}
Does increasing payload size change the real-vs-ML detectability gap?
\end{block}

\begin{block}{RQ3: Embedding-method interaction}
Does method choice (LSB vs DCT) change how carrier origin influences detectability?
\end{block}

\begin{block}{RQ4: Encryption effect}
Does AES-256-CBC payload encryption change detectability, and does this differ by carrier type?
\end{block}
\end{frame}

\begin{frame}{Hypotheses and Decision Criteria}
\small
\renewcommand{\arraystretch}{1.2}
\begin{tabular}{@{}p{1.3cm}p{6.9cm}p{4.4cm}@{}}
\toprule
\textbf{H} & \textbf{Claim} & \textbf{Primary test} \\
\midrule
H1 & Carrier origin affects steganalysis AUC. & Wilcoxon + effect size \\
H2 & Real-vs-ML AUC gap increases with payload. & Spearman trend + ANOVA interaction \\
H3 & Carrier origin effect depends on embedding method. & Two-way ANOVA interaction \\
H4 & Encryption changes AUC and may interact with origin. & Wilcoxon by condition + interaction check \\
\bottomrule
\end{tabular}
\vspace{0.2cm}
\begin{exampleblock}{Common statistical settings}
\small
Primary metric is ROC-AUC with Bonferroni-adjusted alpha (0.05/6 = 0.0083), plus 95\% confidence intervals and Cohen's d.
\end{exampleblock}
\end{frame}

\begin{frame}{Hypotheses: Prior Literature Support}
\footnotesize
\renewcommand{\arraystretch}{1.3}
\begin{tabular}{@{}p{0.55cm}p{5.45cm}p{6.05cm}@{}}
\toprule
\textbf{H} & \textbf{Claim} & \textbf{Supporting reference(s)} \\
\midrule
H1 & Carrier origin affects AUC measured by RS Analysis &
  Fridrich et al.\ (2001); Wang et al.\ (2020); Corvi et al.\ (2023) \\
H2 & Carrier origin affects AUC measured by SRM+FLD &
  Fridrich \& Kodovský (2012); Wang et al.\ (2020); Corvi et al.\ (2023) \\
H3 & Real-vs-ML AUC gap grows monotonically with payload &
  Fridrich et al.\ (2001); Hussain et al.\ (2018); Westfeld \& Pfitzmann (1999) \\
H4 & Embedding method interacts with carrier origin on AUC &
  Chen \& Wornell (2001); Hussain et al.\ (2018); Holub et al.\ (2014) \\
H5 & AES-256 encryption significantly changes steganalysis AUC &
  Westfeld \& Pfitzmann (1999); Fridrich et al.\ (2001); Petitcolas et al.\ (1999) \\
H6 & Encryption effect interacts with carrier type (real vs.\ ML) &
  Wang et al.\ (2020); Corvi et al.\ (2023); De et al.\ (2022) \\
\bottomrule
\end{tabular}
\vspace{0.15cm}
\begin{exampleblock}{\small Why six hypotheses?}
\small
H1/H2 split RQ1 by detector; H5/H6 split RQ4 into main effect vs.\ interaction.
Six tests set the Bonferroni threshold: $\alpha_{\text{adj}} = 0.05/6 \approx 0.0083$.
\end{exampleblock}
\end{frame}

% ============================================================
\section{Chosen Approaches}
% ============================================================
\begin{frame}{Chosen Approach: Factorial Design and Conditions}
\begin{columns}[T]
\begin{column}{0.52\textwidth}
\begin{block}{Design matrix}
\small
\begin{itemize}
\item \textbf{Carrier type (2):} Real, ML-generated
\item \textbf{Embedding (2):} Spatial LSB, frequency-domain DCT-QIM
\item \textbf{Payload (3):} Low, Medium, High
\item \textbf{Detectors (2 main):} RS Analysis, SRM+FLD
\end{itemize}
\end{block}
\end{column}
\begin{column}{0.44\textwidth}
\begin{alertblock}{Sample size}
\small
\begin{itemize}
\item 500 real images
\item 500 ML-generated images
\item 1,000 total carriers
\item Full condition coverage
\end{itemize}
\end{alertblock}
\end{column}
\end{columns}
\vspace{0.1cm}
\begin{exampleblock}{Controlled variable principle}
\small
Carrier origin is treated as the central independent variable; image size/format, embedding pipelines, payload levels, and detector protocols are standardized across conditions.
\end{exampleblock}
\end{frame}

\begin{frame}{Datasets and Preprocessing}
\small
\begin{columns}[T]
\begin{column}{0.48\textwidth}
\begin{block}{Real photographs (500)}
\begin{itemize}
\item RAISE: 250 images (RAW-derived forensic-quality baseline)
\item COCO: 150 images
\item Flickr30k: 100 images
\end{itemize}
\vspace{0.1cm}
Sources: Dang-Nguyen et al. (2015), Lin et al. (2014), Young et al. (2014)
\end{block}
\end{column}
\begin{column}{0.48\textwidth}
\begin{block}{ML-generated images (500)}
\begin{itemize}
\item Stable Diffusion v2.1: 250 images
\item StyleGAN3: 250 images
\item Prompts aligned to COCO/Flickr semantics
\end{itemize}
\vspace{0.1cm}
Sources: Rombach et al. (2022), Karras et al. (2021)
\end{block}
\end{column}
\end{columns}

\begin{alertblock}{Normalization and quality gate}
\small
All images normalized to 512x512 RGB 8-bit PNG. BRISQUE <= 50 filter for generated outputs to exclude low-quality artifacts.
\end{alertblock}
\end{frame}

\begin{frame}{Embedding Methods}
\begin{columns}[T]
\begin{column}{0.49\textwidth}
\begin{block}{LSB substitution (spatial)}
\small
\begin{itemize}
\item PRNG-keyed pixel/channel selection
\item k = 1 for low and medium payload
\item k = 2 for high payload
\item Optional AES-256-CBC payload encryption
\end{itemize}
\end{block}
\end{column}
\begin{column}{0.49\textwidth}
\begin{block}{DCT-QIM (frequency)}
\small
\begin{itemize}
\item 8x8 block DCT per channel
\item Mid-frequency zigzag coefficients (10--54)
\item QIM embedding:
\[
C'_i = \Delta \cdot \mathrm{round}(C_i/\Delta) \pm \Delta/4
\]
\end{itemize}
\end{block}
\end{column}
\end{columns}
\begin{exampleblock}{Method rationale}
\small
LSB and DCT represent the two canonical embedding domains, allowing direct tests of method-origin interactions under controlled payload settings.
\end{exampleblock}
\end{frame}

\begin{frame}{Payload and Encryption Conditions}
\small
\renewcommand{\arraystretch}{1.2}
\begin{tabular}{@{}p{2.3cm}p{2.6cm}p{2.6cm}p{5.0cm}@{}}
\toprule
\textbf{Level} & \textbf{Approx. bpp} & \textbf{LSB setting} & \textbf{Purpose} \\
\midrule
Low & \(\approx 0.08\) & \(k=1\) sparse mask & Near-threshold detectability \\
Medium & \(\approx 0.16\) & \(k=1\) denser mask & Baseline operating point \\
High & \(\approx 0.32\) & \(k=2\) & Stress-test detector sensitivity \\
\bottomrule
\end{tabular}
\vspace{0.2cm}

\begin{columns}[T]
\begin{column}{0.49\textwidth}
\begin{block}{Encryption condition}
\small
Each payload level is tested in:
\begin{itemize}
\item Plain payload mode
\item AES-256-CBC pre-encrypted mode
\end{itemize}
\end{block}
\end{column}
\begin{column}{0.49\textwidth}
\begin{block}{Why include encryption?}
\small
It isolates whether message-bit structure contributes to detectability beyond carrier-level distortion.
\end{block}
\end{column}
\end{columns}
\end{frame}

\begin{frame}{Steganalysis Detectors and Validation Metrics}
\small
\begin{columns}[T]
\begin{column}{0.50\textwidth}
\begin{block}{Detector set}
\small
\begin{itemize}
\item \textbf{RS Analysis} (training-free statistical baseline)
\item \textbf{SRM+FLD} (feature-based classical ML detector)
\item \textbf{\(\chi^2\) attack} as supplementary LSB check
\end{itemize}
\vspace{0.05cm}
References: Fridrich et al. (2001), Westfeld and Pfitzmann (1999), Fridrich and Kodovsky (2012)
\end{block}
\end{column}
\begin{column}{0.46\textwidth}
\begin{block}{Primary metrics}
\small
\begin{itemize}
\item ROC-AUC (primary)
\item Accuracy at Youden's J
\item Equal Error Rate
\item FPR at 5\% FNR
\end{itemize}
\end{block}
\end{column}
\end{columns}

\begin{exampleblock}{Statistical analysis plan}
\small
Two-way ANOVA (carrier x method; payload covariate), Wilcoxon pairwise tests, effect sizes, and Bonferroni-adjusted significance threshold.
\end{exampleblock}
\end{frame}

% ============================================================
\section{Experiments}
% ============================================================
\begin{frame}{Experiment Plan I (RQ1 and RQ2)}
\small
\begin{block}{Exp. 1 -- Carrier origin effect (RQ1)}
\begin{itemize}
\item Apply RS and SRM+FLD across all payload levels and methods.
\item Compare AUC for real vs ML-generated carriers.
\item Decide with Wilcoxon significance and effect size.
\end{itemize}
\end{block}

\begin{block}{Exp. 2 -- Payload sensitivity (RQ2)}
\begin{itemize}
\item Track real-vs-ML AUC gap across Low, Medium, High payload.
\item Test monotonic trend (Spearman) and carrier x payload interaction.
\item Decide whether payload amplifies origin-dependent detectability.
\end{itemize}
\end{block}
\end{frame}

\begin{frame}{Experiment Plan II (RQ3 and RQ4)}
\small
\begin{block}{Exp. 3 -- Method interaction (RQ3)}
\begin{itemize}
\item Run two-way ANOVA on SRM AUC with carrier origin and method factors.
\item Confirm whether the detectability gap depends on LSB vs DCT.
\end{itemize}
\end{block}

\begin{block}{Exp. 4 -- Encryption effect (RQ4)}
\begin{itemize}
\item Compare plain vs AES-encrypted payload AUC per carrier and method.
\item Test whether encryption effect differs by carrier origin.
\end{itemize}
\end{block}

\begin{alertblock}{Interpretation policy}
\small
Null findings remain valid outcomes; all major results reported with confidence intervals and effect sizes.
\end{alertblock}
\end{frame}

% ============================================================
\section{Prototype}
% ============================================================
\begin{frame}{Prototype Status}
\begin{columns}[T]
\begin{column}{0.49\textwidth}
\begin{block}{Vertical prototype (depth)}
\small
Isolated implementation and verification of:
\begin{itemize}
\item LSB embed/extract (BER = 0 on test set)
\item DCT-QIM embed/extract (lossless payload recovery)
\item RS Analysis sanity checks vs expected behavior
\item SRM feature extraction + baseline classifier
\end{itemize}
\end{block}
\end{column}
\begin{column}{0.49\textwidth}
\begin{block}{Horizontal prototype (breadth)}
\small
Integrated end-to-end run on 50 images:
\begin{itemize}
\item 25 real + 25 ML-generated
\item Medium LSB payload condition
\item Pipeline interfaces validated before full scale
\end{itemize}
\end{block}
\end{column}
\end{columns}
\vspace{0.15cm}
{\small\color{umgray}\textbf{Readiness:} Vertical and horizontal checks reduce implementation risk before full 1,000-image execution.}
\end{frame}

% ============================================================
\section{Related Work}
% ============================================================
\begin{frame}{Related Work Landscape}
\small
\begin{block}{Generative steganography}
Prior work often embeds messages during generation itself, including GAN and diffusion pipelines (Hu et al., 2023; Liu et al., 2024; Duan et al., 2020).
\end{block}

\begin{block}{AI-generated carriers}
De et al. (2022) show feasibility with bespoke probabilistic coupling, but without controlled real-vs-ML detectability comparisons.
\end{block}

\begin{block}{Cross-domain and synthetic-image forensics}
Existing studies cover camera-domain shifts and real-vs-synthetic discrimination (Wang et al., 2020; Corvi et al., 2023), but not payload detectability across carrier origin.
\end{block}
\end{frame}

\begin{frame}{Related Work: Our Positioning}
\small
\begin{columns}[T]
\begin{column}{0.49\textwidth}
\begin{block}{What this proposal adds}
\begin{itemize}
\item Controlled real-vs-ML carrier comparison
\item Standardized LSB and DCT embedding
\item Explicit steganalysis outcome reporting
\item Payload and encryption interaction analysis
\end{itemize}
\end{block}
\end{column}
\begin{column}{0.49\textwidth}
\begin{block}{Methodological choice}
\begin{itemize}
\item Classical SRM+FLD over deep detectors
\item Better interpretability for cross-domain behavior
\item Feasible CPU runtime for full study scale
\end{itemize}
\vspace{0.05cm}
References: Fridrich and Kodovsky (2012), Luo et al. (2024)
\end{block}
\end{column}
\end{columns}
\end{frame}

% ============================================================
\section{Curriculum and Planning}
% ============================================================
\begin{frame}{Relation to Curriculum}
\begin{columns}[T]
\begin{column}{0.49\textwidth}
\begin{block}{Cryptography and Steganography}
\small
LSB and DCT-QIM embedding, payload encryption (AES-256-CBC), and detectability-focused reasoning.
\end{block}

\begin{block}{Research Methods}
\small
Factorial design, hypothesis testing, ANOVA/Wilcoxon analysis, effect sizes, and controlled significance correction.
\end{block}
\end{column}
\begin{column}{0.49\textwidth}
\begin{block}{Machine Learning}
\small
Feature-based steganalysis with SRM representations and FLD-style linear classification.
\end{block}

\begin{block}{Algorithm Design and Implementation}
\small
Block-wise transforms, coefficient-level embedding logic, and reproducible experiment orchestration in Python.
\end{block}
\end{column}
\end{columns}
\end{frame}

\begin{frame}{Planning and Milestones}
\small
\begin{columns}[T]
\begin{column}{0.49\textwidth}
\begin{block}{Phase windows}
\begin{itemize}
\item \textbf{Phase 2:} 30 Mar -- 15 May 2026
\item \textbf{Phase 3:} 25 May -- 12 Jun 2026
\end{itemize}
\end{block}

\begin{alertblock}{Milestones}
\small
\begin{itemize}
\item M1 (end Wk2): dataset ready
\item M2 (end Wk3): embedding pipelines verified
\item M3 (end Wk5): experiments complete
\item M4 (end Wk7): report submitted
\end{itemize}
\end{alertblock}
\end{column}
\begin{column}{0.49\textwidth}
\begin{block}{Planned outputs}
\small
\begin{itemize}
\item Curated real and ML-generated datasets
\item Verified LSB/DCT + AES pipelines
\item Detector runs and statistical analysis
\item Final deliverables: slides, poster, paper
\end{itemize}
\end{block}

\begin{block}{Execution strategy}
\small
Parallel workstreams for data, embedding, and analysis with fixed scope after Week 2 to protect timeline and avoid design drift.
\end{block}
\end{column}
\end{columns}
\end{frame}

% ============================================================
\section{Passing Requirements}
% ============================================================
\begin{frame}{Minimal Passing Requirements}
\begin{columns}[T]
\begin{column}{0.49\textwidth}
\begin{block}{Product minimum}
\small
\begin{itemize}
\item Functional LSB and DCT pipelines
\item Plain and AES-encrypted payload modes
\item RS and SRM+FLD evaluated on all 1,000 images
\item Full condition coverage for carrier/method/payload
\end{itemize}
\end{block}
\end{column}
\begin{column}{0.49\textwidth}
\begin{block}{Validation minimum}
\small
\begin{itemize}
\item RQ1 and RQ4 answered with AUC-based significance tests
\item 95\% confidence intervals reported
\item Effect sizes reported for practical significance
\item Null results interpreted explicitly
\end{itemize}
\end{block}
\end{column}
\end{columns}
\vspace{0.2cm}
\begin{exampleblock}{Definition of done}
\small
Minimum deliverable is a reproducible, statistically transparent conclusion on carrier-origin and encryption effects, even if findings are null.
\end{exampleblock}
\end{frame}

% ============================================================
% Closing
% ============================================================
\begin{frame}[plain]
\vfill
\begin{center}
{\color{umorange}\rule{0.32\textwidth}{2pt}}\\[14pt]
{\LARGE\bfseries\color{umdark}Thank you}\\[10pt]
{\large\color{umgray}Questions and Discussion}\\[16pt]
{\small\color{umgray}Midway proposal document: \texttt{docs/proposals/midway\_proposal.tex}}\\[14pt]
{\color{umorange}\rule{0.32\textwidth}{2pt}}
\end{center}
\vfill
\end{frame}

\end{document}
