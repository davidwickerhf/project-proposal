\documentclass[aspectratio=169,11pt]{beamer}

% ---- Minimalist theme ----
\usetheme{default}
\usecolortheme{default}
\setbeamertemplate{navigation symbols}{}
\setbeamertemplate{frametitle continuation}{}
\setbeamertemplate{itemize items}[circle]
\setbeamertemplate{enumerate items}[default]

% ---- Font ----
\usepackage[T1]{fontenc}
\usepackage{tgheros}
\renewcommand{\familydefault}{\sfdefault}
\usepackage{microtype}

% ---- Packages ----
\usepackage{amsmath}
\usepackage{booktabs}
\usepackage{array}
\usepackage{tikz}
\usepackage{xcolor}
\usepackage{hyperref}
\usetikzlibrary{arrows.meta,positioning,calc,fit}

% ---- Maastricht University colors ----
\definecolor{umdark}{HTML}{001C3D}
\definecolor{umorange}{HTML}{E84E10}
\definecolor{umlight}{HTML}{4A90C4}
\definecolor{umgray}{HTML}{6B7280}
\newcommand{\icite}[1]{{\footnotesize\color{umgray!85}#1}}

% ---- Apply colors to Beamer ----
\setbeamercolor{normal text}{fg=umdark}
\setbeamercolor{frametitle}{fg=umdark}
\setbeamercolor{title}{fg=umdark}
\setbeamercolor{subtitle}{fg=umgray}
\setbeamercolor{author}{fg=umgray}
\setbeamercolor{date}{fg=umgray}
\setbeamercolor{institute}{fg=umorange}
\setbeamercolor{itemize item}{fg=umorange}
\setbeamercolor{itemize subitem}{fg=umlight}
\setbeamercolor{enumerate item}{fg=umorange}
\setbeamercolor{block title}{bg=umdark,fg=white}
\setbeamercolor{block body}{bg=umdark!5,fg=umdark}
\setbeamercolor{block title alerted}{bg=umorange,fg=white}
\setbeamercolor{block body alerted}{bg=umorange!8,fg=umdark}
\setbeamercolor{block title example}{bg=umlight,fg=white}
\setbeamercolor{block body example}{bg=umlight!8,fg=umdark}

% ---- Frametitle style ----
\setbeamertemplate{frametitle}{%
  \vspace{0.35cm}%
  \noindent\hspace*{0pt}%
  \parbox{\textwidth}{%
    {\usebeamerfont{frametitle}\usebeamercolor[fg]{frametitle}\insertframetitle}%
    \vspace{2pt}\\%
    {\color{umorange}\rule{\textwidth}{1.2pt}}%
  }%
  \vspace{-2pt}%
}
\setbeamerfont{frametitle}{size=\large,series=\bfseries}
\setbeamersize{text margin left=0.7cm,text margin right=0.7cm}

% ---- Footline ----
\setbeamertemplate{footline}{%
  \hbox{%
    \begin{beamercolorbox}[wd=\paperwidth,ht=2.2ex,dp=1ex]{footline}%
      \hspace{1em}{\scriptsize\color{umgray}Midway Proposal -- 5-Minute Overview}%
      \hfill%
      {\scriptsize\color{umgray}\insertframenumber/\inserttotalframenumber}%
      \hspace{1em}%
    \end{beamercolorbox}%
  }%
}

% ---- Metadata ----
\title{Does the Source of Carrier Image Affect\\Steganographic Detectability?}
\subtitle{Midway Proposal -- 5-Minute Version}
\newcommand{\teamauthors}{Abdul Moiz Akbar \;|\; Malo Coquin \;|\; Daria Gjonbalaj \;|\; Nico Muller-Spath\\Jimena Narvaez del Cid \;|\; David Wicker \;|\; Nikolas Zouros}
\author[Project 2.2 Team]{Project 2.2 Team}
\institute{Department of Advanced Computing Sciences\\Maastricht University}
\date{Project 2.2 \;|\; February 2026}

\begin{document}

% ============================================================
% 1. Opening
% ============================================================
\begin{frame}[plain]
\vfill
\begin{center}
{\color{umorange}\rule{0.42\textwidth}{2pt}}\\[12pt]
{\LARGE\bfseries\color{umdark}\inserttitle}\\[10pt]
{\normalsize\color{umgray}\insertsubtitle}\\[16pt]
{\small\color{umdark}\teamauthors}\\[8pt]
{\small\color{umorange}\insertinstitute}\\[8pt]
{\footnotesize\color{umgray}\insertdate}\\[12pt]
{\color{umorange}\rule{0.42\textwidth}{2pt}}
\end{center}
\vfill
\end{frame}

% ============================================================
% 2. Agenda (content 1/10)
% ============================================================
\begin{frame}{Agenda}
\small
\begin{enumerate}
\item Motivation and problem statement
\item State of the art and research gap
\item Research questions
\item Study design and methods
\item Experiment and evaluation plan
\item Prototype status
\item Research positioning and expected contribution
\item Minimum deliverable threshold
\end{enumerate}
\end{frame}

% ============================================================
% 3. Motivation (content 2/10)
% ============================================================
\section{Motivation and Problem Statement}
\begin{frame}{Motivation and Problem Statement}
\small
\begin{itemize}
\item Steganalysis detectability depends on carrier-image statistics, not only on embedding logic \icite{(Petitcolas et al., 1999; Cheddad et al., 2010; Fridrich \& Kodovsky, 2012).}
\item Most benchmarks assume real camera images; this assumption is weaker as synthetic images become common.
\item Diffusion and GAN images exhibit different statistical traces \icite{(Wang et al., 2020; Corvi et al., 2023).}
\end{itemize}
\vspace{0.1cm}
\begin{alertblock}{Core problem}
Do detectors validated on photographs remain equally effective on ML-generated carriers under identical embedding settings?
\end{alertblock}
\begin{exampleblock}{Study scope}
Controlled \(2\times2\times3\times2\) design, 1{,}000 images (500 real + 500 ML-generated), two embedding methods, three payload levels, two primary detectors.
\end{exampleblock}
\end{frame}

% ============================================================
% 4. State of the art (content 3/10)
% ============================================================
\section{Related Work}
\begin{frame}{State of the Art and Research Gap}
\small
\begin{itemize}
\item Classical post-hoc embedding and detection are well established (LSB, DCT-QIM, RS, \(\chi^2\), SRM).
\item Generative and coverless steganography usually embed during generation, which is a different threat model than ours.
\item Closest prior \icite{(De et al., 2022)} shows feasibility on AI-generated images but not a controlled real-vs-ML detectability comparison under matched LSB/DCT settings.
\item Synthetic-image forensics shows that AI-generated images carry distinct statistical traces, which motivates our carrier-origin hypothesis.
\end{itemize}
\vspace{0.1cm}
\begin{alertblock}{Research gap}
No prior study in our scope isolates \textit{carrier origin} as the key variable while keeping embedding pipeline, payload, and detector settings fixed.
\end{alertblock}
\end{frame}

% ============================================================
% 5. Research Questions (content 4/10)
% ============================================================
\section{Research Questions}
\begin{frame}{Research Questions}
\scriptsize
\begin{columns}[T]
\begin{column}{0.49\textwidth}
\begin{block}{RQ1 (Carrier Origin)}
Is there any effect of carrier-image origin (real vs.\ ML-generated) on detectability of hidden data?
\end{block}
\begin{block}{RQ2 (Payload)}
Does increasing payload size widen the detectability gap between real and ML-generated carriers?
\end{block}
\begin{block}{RQ3 (Encryption)}
Does encrypting payload before embedding make steganography harder or easier to detect, and does origin change this effect?
\end{block}
\end{column}
\begin{column}{0.49\textwidth}
\begin{block}{RQ4 (Embedding Method)}
Do different embedding methods (spatial LSB vs.\ frequency-domain DCT) interact differently with carrier origin in terms of detectability?
\end{block}
\begin{block}{RQ5 (Image Quality)}
Is image quality affected by embedding method and payload size?
\end{block}
\end{column}
\end{columns}
\end{frame}

% ============================================================
% 6. Chosen Approaches I (content 5/10)
% ============================================================
\section{Chosen Approaches}
\begin{frame}{Chosen Approaches: Data and Design}
\small
\begin{columns}[T]
\begin{column}{0.49\textwidth}
\begin{block}{Factorial design}
\begin{itemize}
\item Carrier type: real vs ML-generated
\item Embedding: LSB vs DCT-QIM
\item Payload: low, medium, high
\item Detectors: RS, SRM+FLD
\end{itemize}
\end{block}
\end{column}
\begin{column}{0.49\textwidth}
\begin{block}{Datasets}
\begin{itemize}
\item Real: RAISE (250), COCO (150), Flickr30k (100)
\item ML-generated: Stable Diffusion v2.1 (250), StyleGAN3 (250)
\item All images normalized to 512x512 RGB PNG
\item BRISQUE \(\leq 50\) quality gate for generated images
\end{itemize}
\end{block}
\end{column}
\end{columns}
\end{frame}

% ============================================================
% 7. Chosen Approaches II (content 6/10)
% ============================================================
\begin{frame}{Chosen Approaches: Methods and Validation}
\small
\begin{columns}[T]
\begin{column}{0.49\textwidth}
\begin{block}{Embedding methods}
\begin{itemize}
\item LSB substitution in spatial domain (\(k=1,2\))
\item DCT-QIM in frequency domain (8x8 blocks; zigzag 10--54)
\item Payload optionally encrypted with AES-256-CBC
\end{itemize}
\end{block}
\end{column}
\begin{column}{0.49\textwidth}
\begin{block}{Detection and validation}
\begin{itemize}
\item Primary detectors: RS Analysis, SRM+FLD
\item Supplementary check: \(\chi^2\) attack (LSB)
\item Metrics: ROC-AUC, EER, accuracy at Youden's \(J\)
\item Quality: PSNR, SSIM, FSIM
\end{itemize}
\end{block}
\end{column}
\end{columns}
\end{frame}

% ============================================================
% 8. Experiments (content 7/10)
% ============================================================
\section{Experiments}
\begin{frame}{Experiment and Evaluation Plan}
\scriptsize
\renewcommand{\arraystretch}{1.2}
\begin{tabular}{@{}p{2.2cm}p{9.9cm}@{}}
\toprule
\textbf{Experiment} & \textbf{Focus} \\
\midrule
\textbf{Exp.1 (RQ1)} & Compare real vs.\ ML detectability under matched settings. \\
\textbf{Exp.2 (RQ2)} & Test whether payload level widens the real-vs-ML AUC gap. \\
\textbf{Exp.3 (RQ3)} & Compare plain vs.\ AES-encrypted payload detectability. \\
\textbf{Exp.4 (RQ4)} & Run two-way ANOVA for carrier origin \(\times\) embedding method. \\
\textbf{Exp.5 (RQ5)} & Assess image quality (PSNR/SSIM/FSIM) across conditions. \\
\bottomrule
\end{tabular}
\vspace{0.15cm}
\begin{exampleblock}{Statistical reporting}
We evaluate performance primarily with ROC-AUC and report uncertainty and effect size alongside significance, using a stricter threshold to account for multiple comparisons.
\end{exampleblock}
\end{frame}

% ============================================================
% 9. Prototype (content 8/10)
% ============================================================
\section{Prototype}
\begin{frame}{Prototype Status}
\small
\begin{columns}[T]
\begin{column}{0.49\textwidth}
\begin{block}{Vertical prototype}
\begin{itemize}
\item LSB embedding/extraction
\item DCT-QIM embedding
\item RS Analysis
\item SRM feature extraction/classification
\end{itemize}
\end{block}
\end{column}
\begin{column}{0.49\textwidth}
\begin{block}{Horizontal prototype}
\begin{itemize}
\item Integrated run on 50 images
\item 25 real + 25 ML-generated
\item Medium-payload pipeline check
\item Interfaces validated before full run
\end{itemize}
\end{block}
\end{column}
\end{columns}
\end{frame}

% ============================================================
% 10. Positioning (content 9/10)
% ============================================================
\section{Positioning and Contribution}
\begin{frame}{Research Positioning and Expected Contribution}
\small
\begin{columns}[T]
\begin{column}{0.49\textwidth}
\begin{block}{What we are trying to do}
\begin{itemize}
\item Quantify whether ML-generated carriers change detectability.
\item Compare real vs ML under identical LSB/DCT embedding conditions.
\item Measure effects of payload and encryption, not just feasibility.
\end{itemize}
\end{block}
\end{column}
\begin{column}{0.49\textwidth}
\begin{block}{How it fits current research}
\begin{itemize}
\item Bridges steganalysis and synthetic-image forensics.
\item Uses standardized classical baselines for comparability.
\item Produces direct evidence on whether existing detectors transfer to mixed real/synthetic traffic.
\end{itemize}
\end{block}
\end{column}
\end{columns}
\end{frame}

% ============================================================
% 11. Passing Requirements (content 10/10)
% ============================================================
\section{Minimum Deliverable}
\begin{frame}{Minimum Deliverable Threshold}
\small
\begin{block}{Approach threshold}
\begin{itemize}
\item At least one encryption algorithm
\item At least two embedding methods:
\item one spatial-domain and one frequency-domain
\end{itemize}
\end{block}
\begin{block}{Research threshold}
At minimum, answer:
\begin{itemize}
\item RQ3 (Encryption)
\item RQ4 (Embedding-method interaction)
\end{itemize}
\end{block}
\end{frame}

% ============================================================
% 12. Closing
% ============================================================
\begin{frame}[plain]
\vfill
\begin{center}
{\color{umorange}\rule{0.32\textwidth}{2pt}}\\[14pt]
{\LARGE\bfseries\color{umdark}Thank you}\\[10pt]
{\large\color{umgray}Questions and Discussion}\\[16pt]
{\small\color{umgray}Document: \texttt{docs/proposals/midway\_proposal\_final.tex}}\\[14pt]
{\color{umorange}\rule{0.32\textwidth}{2pt}}
\end{center}
\vfill
\end{frame}

\end{document}
